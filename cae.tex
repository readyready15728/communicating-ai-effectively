\documentclass{beamer}
\usepackage[utf8]{inputenc}
\usepackage[T1]{fontenc}
\usepackage[sfdefault]{FiraSans}
\usepackage{verbatim}
\usepackage{graphicx}
\usetheme{Antibes}
\usecolortheme{dolphin}

\title{Communicating AI Effectively}
\author{Lynn Bradshaw}
\date{10 August 2019}

\begin{document}
  \begin{frame}
    \maketitle
  \end{frame}

  \begin{frame}
    \frametitle{Outline}
    \tableofcontents[pausesections]
  \end{frame}

  \section{Shameless Advertisement of the Presentation Medium}
  \section{Don't Just Take My Word for It}
  \section{Taxonomy of AI and Its Uses}
  \section{Assessing Realisticness of Applications}
  \section{The Geography of AI Development}
  \section{Impacts on Employment}
  \section{Future of Human-AI Interaction}
  \section{Impact on Philosophy and Religion}
  % Mention kami before 40K
  \section{Impact on Ethics}
  \section{AI and Statistical Literacy}
  \section{Recommended Reading}
  
  \begin{comment}
Subject: Communicating AI Effectively

Possible topics (will likely have to pare them down due to time limitations):

* Taxonomy of AI and its uses:

** Different methods, both at the level of categories (e.g. supervised
learning), and concrete methods (e.g. k-NN). These are most popular these days
but can also be contrasted with traditional methods such as tree search that
dominate texts like Artificial Intelligence: A Modern Approach.

** Applications of these methods (NLP, image classification, computer
opponents, recommender systems, etc.).

* What's realistic and what's not realistic. I'd probably only call anything
involving something like hypercomputation impossible but a lot of things are
just really most likely not going to happen. This would include anything like
Skynet. (This has long been difficult even for luminaries in the field. For
example when Logic Theorist was created and came up with better proofs for
theorems in Principia Mathematica (the Russell and Whitehead one, not the
Newton one) than the human authors, this led to enthusiastic thinking that
human-level or better artificial intelligence would appear in a few decades.
This did not happen.) Of course for all but one of you I am now beating a dead
horse but the most likely path to the superhuman is an enhanced human rather
than something built from scratch. This may or may not include integration of
artificially intelligent components with the human body.

* Assessment of strengths in development and use of AI technology by country /
region. Primarily this means comparing the USA and PRC. Japan (as well as the
broader Jakota triangle of Japan, ROK and Taiwan), Western Europe, India and
Israel also merit some attention.

* How structural unemployment is likely to play out. ("You will almost
certainly be far safer as a roofer than as a legal researcher...")

* Impact of human-AI interaction in the future. (Japan being a very good
example.)

* Impact of AI and other GRIN (genetic, robotic, information technology,
nanotechnology) technologies on religions and other belief systems. "Dataism"
as put forward by Yuval Noah Harari leading example. Possible compatibility of
traditional religious thinking with mechanistic ideas of the mind typical in
cognitive science assessed as well: Descartes and Hobbes had mechanistic ideas
of the mind and at least one of them (Descartes) was almost definitely a
sincere believer. (Hobbes may have been a closeted atheist.) Japanese notion
of kami where entities not usually considered to have a spirit by other people
also highly relevant—priority of making robots human-like in Japan not likely
coincidental. Fictional examples like the Machine Cult from the card game Star
Realms and Meklar species from the Master of Orion games also worth mentioning
because they reflect on what we think might happen someday.

* Recommended reading / people: both of Yuval Noah Harari's future-oriented
books and The Master Algorithm by Pedro Domingos stand out in my mind
particularly. I also have a high opinion of The Deep Learning Revolution by
Terrence J. Sejnowski. Kinda ambivalent about Weapons of Math Destruction by
Cathy O'Neill because I think the ideological slant ends up in points that
aren't very good in some places but it's still worth a read. Heard good things
about Kai-Fu Lee's book but haven't read it yet. I've read and heard a lot
from Rodney Brooks and generally he's very on point as is another roboticist,
Rolf Pfeifer. Also look into the ideas of J.C.R. Licklider, who was ahead of
the curve by decades—Man-Computer Symbiosis was published in 1960.

* Statistical literacy—most of what we now call "AI" is heavily dependent on
statistics. More technical but includes things like knowing that a p-value is
a conditional probability, statistical vs. practical significance, why one
shouldn't do a whole bunch of significance tests without correcting for
multiple comparisons, frequentism vs. Bayesianism etc.

* Machine ethics: who will get run over by an autonomous car if there is no
other option? Goes into how Isaac Asimov's famous Laws of Robotics don't
really work. (He knew this and often exploited it in his writing.) Military is
of course another issue. There's an interesting book about this, Governing
Lethal Behavior in Autonomous Robots by Ronald C. Arkin. The methods proposed
are kind of high-level and hand-wavey but he does a lot to integrate
traditional Western ethics of war with current prospects of warfighting
machines. Other legal issues.
  \end{comment}
\end{document}
